\documentclass[a4paper, 11pt]{article}

\usepackage[ngerman, english]{babel}
\usepackage[latin9]{inputenc}
\usepackage{color}
\usepackage{graphicx}
\usepackage{multirow}
\usepackage[verbose]{hyperref}
\usepackage[verbose]{placeins}
\usepackage{amsmath,mathtools}
\usepackage{amssymb}
\usepackage{bm}
\usepackage[super]{nth}
\usepackage{url}
\usepackage{subfig}
\usepackage{wasysym}
\usepackage{slashed}
\usepackage{fancyhdr}
\usepackage{lipsum}
\usepackage[top=1.0in, bottom=0.9in, left=0.8in, right=0.8in]{geometry}
\usepackage{datetime}
\usepackage{verbatim}
\usepackage{siunitx}
\usepackage{feynmp}
\usepackage{fancybox}
\usepackage{empheq}
\usepackage{braket}
\usepackage{multicol}

\sisetup{output-exponent-marker=\ensuremath{\mathrm{e}}}

% literature
\usepackage[style=phys,maxbibnames=6,articletitle=true,biblabel=brackets,%
  chaptertitle=true,pageranges=true,eprint=true, natbib=true, block=space, backend=bibtex, sorting=none]{biblatex}

% interpret all .1 / .2 files (from feynmp) as eps (which they really are...)
% to update the graphics-file, run >> mpost feyn <<
\DeclareGraphicsRule{*}{mps}{*}{}

\bibliography{db}
\defbibheading{literature}{\paragraph{Bibliography:}}

\makeatletter
\renewcommand\@maketitle{%
  \begin{center}
    \begin{minipage}{\textwidth}
      \centering
          {\fontfamily{qtm}\selectfont
            \noindent\Large \textbf{\@title}\\[1em]
            \noindent\large \@author\\[0.4em]
            \noindent\small\@date
          }
    \end{minipage}
  \end{center}
}
\makeatother

\title{STIR and Tensorflow}
\author{Philipp Windischhofer\thanks{philipp.windischhofer@cern.ch, philipp.windischhofer@gmail.com}}
\date{\today}

% allows equations to be page-broken
\allowdisplaybreaks

\begin{document}
\begin{fmffile}{feyn}

  \maketitle

  \begin{abstract}
    \noindent This document provides a bit more in-depth and technical information regarding the integration of Tensorflow into STIR. It should give enough hints and tips such as to ease further development and extension of the ideas presented here.
  \end{abstract}

  \section{The overall setup}
  take motivation from presentation, put time spent in forward projection etc.

  \section{Setting up Tensorflow}
  how to link the shared library with CMake, link to the guide, try different CUDA compilers
  
  \section{Creating the graphs}
  python scripts, describe functionality and basic design rules, timeline for debugging

  \section{Bringing Tensorflow into STIR}
  use the changed CMake files, don't reconfigure, change C++11 standard for recon buildblock

  \section{Computing matrix elements with Tensorflow}
  put more details for sdf, ray marching algorithm \& scheduling of matrix elements, how to activate TF support

  \section{Class overview}
  TFRayTracer, ProjMatrixByBinUsingRayTracingTF, describe repository contents, tensorboard for debugging etc...

  \section{Futher information}


\end{fmffile}
\end{document}


